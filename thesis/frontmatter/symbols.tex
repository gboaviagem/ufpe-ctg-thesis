% \pdfbookmark[0]{List of symbols}{symbols}
% \chapter*{List of symbols}
% \label{sec:symbols}
% %\vspace*{-10mm}


\item[$ \mathbf{A} $] Matriz de adjacência ponderada do grafo.
\item[$ \mathbf{L} $] Matriz Laplaciana do grafo.
\item[$ \mathbf{\Lambda}, \mathbf{J} $] Respectivamente a matriz de autovalores (se existir) e a matriz de Jordan da matriz de adjacência do grafo.
\item[$ \mathbf{V} $] A matriz de autovetores (possivelmente generalizada) da matriz de adjacência do grafo.
\item[$\mathcal{R}e \{ x \}$] Parte real do número complexo $x$.
\item[$\mathcal{I}m \{ x \}$] Parte imaginária do número complexo $x$.
\item[$ \overline{x} $] Conjugado do número complexo (ou quaterniônico) $x$. Se vetores ou matrizes forem usados em vez de $x$, a conjugação é realizada em cada uma de suas entradas.
\item[$\mathbf{M}^T$] Transposta da matriz $\mathbf{M}$.
\item[$\mathbf{M}^H$] Transposta conjugada da matriz $\mathbf{M}$.
\item[$\mathbb{R}$, $\mathbb{C}$ e $\mathbb{H}$] Respectivamente, o conjunto dos números reais, complexos e (hamiltonianos) quaterniônicos. Por associação, também podemos nos referir ao respectivo \emph{skew field}.